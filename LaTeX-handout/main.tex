\documentclass{article}

\usepackage{amsmath, amsfonts, mathrsfs}

\usepackage{geometry}
\geometry {
    a4paper,
    left = 4cm,
    right = 4cm,
    top = 3cm,
    bottom = 3cm
}

\title{\huge \LaTeX \space handout}
\author{tinyynoob}
\date{\large \today}

% I am comment

\begin{document}

\begin{titlepage}
    \maketitle
    \vfill
    \hfill author: tinyynoob

    \hfill contact: sadpotato1920@gmail.com
    \thispagestyle{empty}
\end{titlepage}

\section*{Preface}

LaTeX is an excellent tool for word processing, especially for those containing mathematical symbols or equations.

This handout provides some examples for common-used syntax in LaTeX. You can download such as TeX Live to use LaTeX at local side. Or you may use it online on Overleaf, which provides convenience.

This handout is written at CCU during event "Pick up student! Pick up teacher!".
\newpage

\tableofcontents
\newpage

\section{Title}

section declares title

\subsection{Sub title}

Lorem Ipsum is simply dummy text of the printing and typesetting industry.

\subsubsection{Level 3 title}

Lorem Ipsum has been the industry's standard dummy text ever since the 1500s, when an unknown printer took a galley of type and scrambled it to make a type specimen book.

\subsubsection{Another level 3 title}

It has survived not only five centuries, but also the leap into electronic typesetting, remaining essentially unchanged. 

\subsection{Paragraph, Newline and Space}

Hello, world! Hello, world! \newline
Hello, world!                Hello, world! \newline
Hello, world! \\
Hello, world!
Hello, world! \par
Hello, world! \\
Hello, world!

\subsection{Indents}

\subsubsection{Normal}

The standard chunk of Lorem Ipsum used since the 1500s is reproduced below for those interested. \par
Sections 1.10.32 and 1.10.33 from "de Finibus Bonorum et Malorum" by Cicero are also reproduced in their exact original form, accompanied by English versions from the 1914 translation by H. Rackham.


\subsubsection{No indents}

The standard chunk of Lorem Ipsum used since the 1500s is reproduced below for those interested. \par
Sections 1.10.32 and 1.10.33 from "de Finibus Bonorum et Malorum" by Cicero are also reproduced in their exact original form, accompanied by English versions from the 1914 translation by H. Rackham.

\subsubsection{flushleft}

\begin{flushleft}
The standard chunk of Lorem Ipsum used since the 1500s is reproduced below for those interested. \par
Sections 1.10.32 and 1.10.33 from "de Finibus Bonorum et Malorum" by Cicero are also reproduced in their exact original form, accompanied by English versions from the 1914 translation by H. Rackham.
\end{flushleft}

\subsubsection{flushright}

\begin{flushright}
The standard chunk of Lorem Ipsum used since the 1500s is reproduced below for those interested. \par
Sections 1.10.32 and 1.10.33 from "de Finibus Bonorum et Malorum" by Cicero are also reproduced in their exact original form, accompanied by English versions from the 1914 translation by H. Rackham.
\end{flushright}

\subsection{fill}

I am at beginning.

Normal position

\hfill hfill to here

\vfill
\begin{center}
    I am at bottom of this page, by vfill.
\end{center}
\newpage

\section{Objects}

\subsection{Ordered List}

\begin{enumerate}
    \item
    \begin{enumerate}
        \item int
        \item long
    \end{enumerate}
    \item floating types
    \begin{enumerate}
        \item float
        \item double
    \end{enumerate}
\end{enumerate}

\subsection{Unordered List}

\subsubsection{normal}

\begin{itemize}
    \item apple
    \item watermelon
    \item pear
\end{itemize}

\subsubsection{custom}

\begin{itemize}
    \item[-] abcd
    \item[$>$] I look like terminal.
    \item[] ijkl
    \item mnop
    \item[$\square$] qrst
\end{itemize}

\subsection{Tabular}

\subsubsection{example 1}
\begin{tabular}{c|c c}
    cell1 & cell2 & cell3 \\
    cell4 & cell5 & cell6 \\  
    cell7 & cell8 & cell9    
\end{tabular}

\subsubsection{example 2}
\begin{tabular}{l|c|r}
    left & mid & right \\
    \hline
    celllllllll4 & cell5 & cellllllll6 \\  
    \hline
    celllllllll7 & cell8 & cellllllll9    
\end{tabular}

\newpage
\section{Math}

There are two modes for math in \LaTeX .

$x = 3\div 5 = \frac{3}{5}$ may be inlined in words, or be displayed themselves: 

$$
x = 3\div 5 = \frac{3}{5}
$$

\[
x = \quad 3\div 5
\]

Don't forget to import package {\it amsmath} at front!

Inline mathematical expression  $1+3+5+7+9+11+13+15+17+19+21+23+25+27+29+31+33+35+37+39+41+43+45+47+49+51+53+55+57+59+61+63+64+67+69$ would be wrapped automatically.

\subsection{Symbols}

The display of the symbols are very easy.

$x$, $y$, $z$, $\alpha$, $\beta$, $\phi$, $\Phi$, $\omega$, $\Omega$

$\exists$, $\forall$, $>$, $\geq$, $\leq$, $\gg$, $\pm$, $\mp$, $/$, $\backslash$

\subsection{Subscripts}

$$
x^3 \quad x^y \quad \pi^2 \quad x^20 \quad x^{20}
$$

$$
x_3 \quad x_y \quad \delta_2 \quad x_20 \quad x_{20}
$$

\[
x_5^7 \quad x_5^{79} \quad x_{52}^7 \quad x_{52}^{79} \quad x_i^2 \quad x_{i^2}
\]

$$
\lim_{x\to 0^+}\frac{1}{x} = +\infty
$$

$$
v' \quad \vec{v} \quad \hat{v} \quad \bar{v} \quad \tilde{v}
$$

\subsection{Parenthesis}

\subsection{Math font style}

\begin{itemize}
    \item E 
    $$
    \text{E} \quad E \quad \mathrm{E} \quad \mathbf{E} \quad \mathcal{E} \quad \mathbb{E} \quad \mathscr{E}
    $$
    \item L
    $$
    \text{L} \quad L \quad \mathrm{L} \quad \mathbf{L} \quad \mathcal{L} \quad \mathbb{L} \quad \mathscr{L}
    $$
\end{itemize}


\subsection{cases}

\[
\mathrm{sgn}(x) = \begin{cases}
    1 & x > 0 \\
    0 & x = 0 \\
    -1 & x < 1
\end{cases}
\]

\subsection{matrix}

\[
\begin{matrix}
    a & b \\
    c & d
\end{matrix}
\]

\[
\det\begin{bmatrix}
    a & b \\
    c & d
\end{bmatrix} = 
\begin{vmatrix}
    a & b \\
    c & d
\end{vmatrix}
= ad - bc
\]

\[
\begin{pmatrix}
    1 & 2 & 0\\
    3 & 4 & 0
\end{pmatrix}_{2\times 3}^T
=
\begin{pmatrix}
    1 & 3\\
    2 & 4\\
    0 & 0
\end{pmatrix}
\]

\subsection{align}

\begin{align}
    y &= \int_1^5 3x\, dx \\
     &= \left. \frac{3}{2}x^2 \right]_1^5 \\
     &= \frac{3}{2} \cdot (25 - 1) = 36
\end{align}

\begin{align}
    \cos(x) &= \frac{\mathrm{d}\sin x}{\mathrm{d}x} \\
     &= \sum_{n=0}^\infty \frac{(-1)^n}{(2n)!}\cdot x^{2n - 1} \\
     &= 1 - \frac{x^2}{2!} + \frac{x^4}{4!} - \frac{x^6}{6!} + \cdots
\end{align}

\subsection{substack}




\end{document}
